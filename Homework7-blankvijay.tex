\documentclass[]{article}
\usepackage{lmodern}
\usepackage{amssymb,amsmath}
\usepackage{ifxetex,ifluatex}
\usepackage{fixltx2e} % provides \textsubscript
\ifnum 0\ifxetex 1\fi\ifluatex 1\fi=0 % if pdftex
  \usepackage[T1]{fontenc}
  \usepackage[utf8]{inputenc}
\else % if luatex or xelatex
  \ifxetex
    \usepackage{mathspec}
  \else
    \usepackage{fontspec}
  \fi
  \defaultfontfeatures{Ligatures=TeX,Scale=MatchLowercase}
\fi
% use upquote if available, for straight quotes in verbatim environments
\IfFileExists{upquote.sty}{\usepackage{upquote}}{}
% use microtype if available
\IfFileExists{microtype.sty}{%
\usepackage{microtype}
\UseMicrotypeSet[protrusion]{basicmath} % disable protrusion for tt fonts
}{}
\usepackage[margin=1in]{geometry}
\usepackage{hyperref}
\hypersetup{unicode=true,
            pdftitle={ST 411/511 Homework 7},
            pdfauthor={Vijay Tadimeti},
            pdfborder={0 0 0},
            breaklinks=true}
\urlstyle{same}  % don't use monospace font for urls
\usepackage{color}
\usepackage{fancyvrb}
\newcommand{\VerbBar}{|}
\newcommand{\VERB}{\Verb[commandchars=\\\{\}]}
\DefineVerbatimEnvironment{Highlighting}{Verbatim}{commandchars=\\\{\}}
% Add ',fontsize=\small' for more characters per line
\usepackage{framed}
\definecolor{shadecolor}{RGB}{248,248,248}
\newenvironment{Shaded}{\begin{snugshade}}{\end{snugshade}}
\newcommand{\KeywordTok}[1]{\textcolor[rgb]{0.13,0.29,0.53}{\textbf{#1}}}
\newcommand{\DataTypeTok}[1]{\textcolor[rgb]{0.13,0.29,0.53}{#1}}
\newcommand{\DecValTok}[1]{\textcolor[rgb]{0.00,0.00,0.81}{#1}}
\newcommand{\BaseNTok}[1]{\textcolor[rgb]{0.00,0.00,0.81}{#1}}
\newcommand{\FloatTok}[1]{\textcolor[rgb]{0.00,0.00,0.81}{#1}}
\newcommand{\ConstantTok}[1]{\textcolor[rgb]{0.00,0.00,0.00}{#1}}
\newcommand{\CharTok}[1]{\textcolor[rgb]{0.31,0.60,0.02}{#1}}
\newcommand{\SpecialCharTok}[1]{\textcolor[rgb]{0.00,0.00,0.00}{#1}}
\newcommand{\StringTok}[1]{\textcolor[rgb]{0.31,0.60,0.02}{#1}}
\newcommand{\VerbatimStringTok}[1]{\textcolor[rgb]{0.31,0.60,0.02}{#1}}
\newcommand{\SpecialStringTok}[1]{\textcolor[rgb]{0.31,0.60,0.02}{#1}}
\newcommand{\ImportTok}[1]{#1}
\newcommand{\CommentTok}[1]{\textcolor[rgb]{0.56,0.35,0.01}{\textit{#1}}}
\newcommand{\DocumentationTok}[1]{\textcolor[rgb]{0.56,0.35,0.01}{\textbf{\textit{#1}}}}
\newcommand{\AnnotationTok}[1]{\textcolor[rgb]{0.56,0.35,0.01}{\textbf{\textit{#1}}}}
\newcommand{\CommentVarTok}[1]{\textcolor[rgb]{0.56,0.35,0.01}{\textbf{\textit{#1}}}}
\newcommand{\OtherTok}[1]{\textcolor[rgb]{0.56,0.35,0.01}{#1}}
\newcommand{\FunctionTok}[1]{\textcolor[rgb]{0.00,0.00,0.00}{#1}}
\newcommand{\VariableTok}[1]{\textcolor[rgb]{0.00,0.00,0.00}{#1}}
\newcommand{\ControlFlowTok}[1]{\textcolor[rgb]{0.13,0.29,0.53}{\textbf{#1}}}
\newcommand{\OperatorTok}[1]{\textcolor[rgb]{0.81,0.36,0.00}{\textbf{#1}}}
\newcommand{\BuiltInTok}[1]{#1}
\newcommand{\ExtensionTok}[1]{#1}
\newcommand{\PreprocessorTok}[1]{\textcolor[rgb]{0.56,0.35,0.01}{\textit{#1}}}
\newcommand{\AttributeTok}[1]{\textcolor[rgb]{0.77,0.63,0.00}{#1}}
\newcommand{\RegionMarkerTok}[1]{#1}
\newcommand{\InformationTok}[1]{\textcolor[rgb]{0.56,0.35,0.01}{\textbf{\textit{#1}}}}
\newcommand{\WarningTok}[1]{\textcolor[rgb]{0.56,0.35,0.01}{\textbf{\textit{#1}}}}
\newcommand{\AlertTok}[1]{\textcolor[rgb]{0.94,0.16,0.16}{#1}}
\newcommand{\ErrorTok}[1]{\textcolor[rgb]{0.64,0.00,0.00}{\textbf{#1}}}
\newcommand{\NormalTok}[1]{#1}
\usepackage{graphicx,grffile}
\makeatletter
\def\maxwidth{\ifdim\Gin@nat@width>\linewidth\linewidth\else\Gin@nat@width\fi}
\def\maxheight{\ifdim\Gin@nat@height>\textheight\textheight\else\Gin@nat@height\fi}
\makeatother
% Scale images if necessary, so that they will not overflow the page
% margins by default, and it is still possible to overwrite the defaults
% using explicit options in \includegraphics[width, height, ...]{}
\setkeys{Gin}{width=\maxwidth,height=\maxheight,keepaspectratio}
\IfFileExists{parskip.sty}{%
\usepackage{parskip}
}{% else
\setlength{\parindent}{0pt}
\setlength{\parskip}{6pt plus 2pt minus 1pt}
}
\setlength{\emergencystretch}{3em}  % prevent overfull lines
\providecommand{\tightlist}{%
  \setlength{\itemsep}{0pt}\setlength{\parskip}{0pt}}
\setcounter{secnumdepth}{0}
% Redefines (sub)paragraphs to behave more like sections
\ifx\paragraph\undefined\else
\let\oldparagraph\paragraph
\renewcommand{\paragraph}[1]{\oldparagraph{#1}\mbox{}}
\fi
\ifx\subparagraph\undefined\else
\let\oldsubparagraph\subparagraph
\renewcommand{\subparagraph}[1]{\oldsubparagraph{#1}\mbox{}}
\fi

%%% Use protect on footnotes to avoid problems with footnotes in titles
\let\rmarkdownfootnote\footnote%
\def\footnote{\protect\rmarkdownfootnote}

%%% Change title format to be more compact
\usepackage{titling}

% Create subtitle command for use in maketitle
\newcommand{\subtitle}[1]{
  \posttitle{
    \begin{center}\large#1\end{center}
    }
}

\setlength{\droptitle}{-2em}

  \title{ST 411/511 Homework 7}
    \pretitle{\vspace{\droptitle}\centering\huge}
  \posttitle{\par}
  \subtitle{Due on March 6}
  \author{Vijay Tadimeti}
    \preauthor{\centering\large\emph}
  \postauthor{\par}
      \predate{\centering\large\emph}
  \postdate{\par}
    \date{Winter 2019}


\begin{document}
\maketitle

\section{Instructions}\label{instructions}

This assignment is due by 11:59 PM, March 6, 2019 on Canvas.

\textbf{You should submit your assignment as either a PDF, Word, or html
document, which you can compile (should you choose -- recommended) from
the provided .Rmd (R Markdown) template.} Please include your code.

\section{Problems (25 points total)}\label{problems-25-points-total}

\subsection{Question 1}\label{question-1}

\subsubsection{\texorpdfstring{(a) (2 points) In comparing 10 groups, a
researcher notices that the sample mean of group 7 is the largest and
the sample mean of group 3 is the smallest. The researcher then decides
to test the hypothesis that \(\mu_7-\mu_3=0\). Why should a multiple
comparison procedure be used even though there is only one comparison
being
made?}{(a) (2 points) In comparing 10 groups, a researcher notices that the sample mean of group 7 is the largest and the sample mean of group 3 is the smallest. The researcher then decides to test the hypothesis that \textbackslash{}mu\_7-\textbackslash{}mu\_3=0. Why should a multiple comparison procedure be used even though there is only one comparison being made?}}\label{a-2-points-in-comparing-10-groups-a-researcher-notices-that-the-sample-mean-of-group-7-is-the-largest-and-the-sample-mean-of-group-3-is-the-smallest.-the-researcher-then-decides-to-test-the-hypothesis-that-mu_7-mu_30.-why-should-a-multiple-comparison-procedure-be-used-even-though-there-is-only-one-comparison-being-made}

Here we see that we have 10 groups, and for the difference between the
smallest and largest averages is possible only if there is a large
difference between them. We should use a multiple comparison for making
sure that the statistical measure of uncertainty is equal to the one
that is appropriate for comparing every mean to every other mean.

\subsubsection{\texorpdfstring{b) (2 points) When choosing coefficients
for a contrast, does the choice of \(\{C_1, C_2, \ldots, C_I\}\) give a
different \(t\)-statistic than the choice of
\(\{4C_1, 4C_2, \ldots, 4C_I\}\)? Explain why or why
not.}{b) (2 points) When choosing coefficients for a contrast, does the choice of \textbackslash{}\{C\_1, C\_2, \textbackslash{}ldots, C\_I\textbackslash{}\} give a different t-statistic than the choice of \textbackslash{}\{4C\_1, 4C\_2, \textbackslash{}ldots, 4C\_I\textbackslash{}\}? Explain why or why not.}}\label{b-2-points-when-choosing-coefficients-for-a-contrast-does-the-choice-of-c_1-c_2-ldots-c_i-give-a-different-t-statistic-than-the-choice-of-4c_1-4c_2-ldots-4c_i-explain-why-or-why-not.}

It does not give a different t statistic. We observe that the parameter
changes from γ to 4γ, hence the estimate changes from g to 4g. We also
see that the standard error changes SE(4g) = 4SE(g). Hence we observe
that the t ratio is not changed. Hence we can multiply the coefficients
by one common factor to make all the coefficients into integers if
needed.

\subsection{\texorpdfstring{Question 2 (Modified from \emph{Sleuth}
6.17)}{Question 2 (Modified from Sleuth 6.17)}}\label{question-2-modified-from-sleuth-6.17}

The relative head length (RHL) is measured for adders (a type of snake)
on the Swedish mainland and on groups of islands in the Baltic Sea.
Relative head length is adjusted for overall body length, determined
separately for males and females. The data are below and additionally
you know that the pooled estimate of standard deviation of the RHL
measurements was 11.72 based on 230 degrees of freedom.

\begin{Shaded}
\begin{Highlighting}[]
\NormalTok{adder <-}\StringTok{ }\KeywordTok{data.frame}\NormalTok{(}\DataTypeTok{Locality =} \KeywordTok{c}\NormalTok{(}\StringTok{"Uppsala"}\NormalTok{, }\StringTok{"In-Fredeln"}\NormalTok{, }\StringTok{"Inre Hammnskar"}\NormalTok{, }\StringTok{"Norrpada"}\NormalTok{,}
                                 \StringTok{"Karringboskar"}\NormalTok{, }\StringTok{"Angskar"}\NormalTok{, }\StringTok{"SvenskaHagarna"}\NormalTok{), }
                    \DataTypeTok{SampleSize =} \KeywordTok{c}\NormalTok{(}\DecValTok{21}\NormalTok{, }\DecValTok{34}\NormalTok{, }\DecValTok{20}\NormalTok{, }\DecValTok{25}\NormalTok{, }\DecValTok{7}\NormalTok{, }\DecValTok{82}\NormalTok{, }\DecValTok{48}\NormalTok{), }
                    \DataTypeTok{meanRHL =} \KeywordTok{c}\NormalTok{(}\OperatorTok{-}\FloatTok{6.98}\NormalTok{, }\OperatorTok{-}\FloatTok{4.24}\NormalTok{, }\OperatorTok{-}\FloatTok{2.79}\NormalTok{, }\FloatTok{2.22}\NormalTok{, }\FloatTok{1.27}\NormalTok{, }\FloatTok{1.88}\NormalTok{, }\FloatTok{4.98}\NormalTok{))}
\end{Highlighting}
\end{Shaded}

Consider the question: ``Is the average of the mean relative head
lengths for snakes on the Swedish mainland equal to the average of the
mean relative head lengths for snakes on islands in the Baltic Sea?''
Uppsala is the mainland, and the other six localities refer to islands
in the Baltic Sea.

\subsubsection{(a) (3 points) Give the coefficients for the linear
combination you would use to test this question, and state the null
hypothesis you would be testing using statistical
notation.}\label{a-3-points-give-the-coefficients-for-the-linear-combination-you-would-use-to-test-this-question-and-state-the-null-hypothesis-you-would-be-testing-using-statistical-notation.}

We observe that there are 7 variables given which are Uppsala,
In-Fredeln, Inre Hammnskar, Norrpada, Karringboskar,
Angskar,SvenskaHagarna which we can consider as: C1, C2, C3, C4, C5, C6,
C7 and their corresponding values are C1 = 1, followed by C2 = C3 = C4 =
C5 = C6 = C7 = -1/6. All the coefficients should sum to 0. the null
hypothesis is: gamma = 1μ1−1/6(μ2+μ3+μ4+μ5+μ6+μ7)

\subsubsection{\texorpdfstring{(b) (4 points) What is the
\(t\)-statistic for testing the hypothesis in part (a)? Please include
in your answer your computed values of \(g\) and the standard error of
\(g\).}{(b) (4 points) What is the t-statistic for testing the hypothesis in part (a)? Please include in your answer your computed values of g and the standard error of g.}}\label{b-4-points-what-is-the-t-statistic-for-testing-the-hypothesis-in-part-a-please-include-in-your-answer-your-computed-values-of-g-and-the-standard-error-of-g.}

\begin{Shaded}
\begin{Highlighting}[]
\NormalTok{C1 =}\StringTok{ }\DecValTok{1}
\NormalTok{C2 =}\StringTok{ }\NormalTok{C3 =}\StringTok{ }\NormalTok{C4 =}\StringTok{ }\NormalTok{C5 =}\StringTok{ }\NormalTok{C6 =}\StringTok{ }\NormalTok{C7 =}\StringTok{ }\OperatorTok{-}\DecValTok{1}\OperatorTok{/}\DecValTok{6} \CommentTok{#so that they sum to 0.}
\NormalTok{g =}\StringTok{ }\NormalTok{C1 }\OperatorTok{*}\StringTok{ }\OperatorTok{-}\FloatTok{6.98} \OperatorTok{+}\StringTok{ }\NormalTok{(}\OperatorTok{-}\DecValTok{1}\OperatorTok{/}\DecValTok{6} \OperatorTok{*}\StringTok{ }\NormalTok{(}\OperatorTok{-}\FloatTok{4.24} \OperatorTok{+}\StringTok{ }\OperatorTok{-}\FloatTok{2.79} \OperatorTok{+}\StringTok{ }\FloatTok{2.22} \OperatorTok{+}\StringTok{ }\FloatTok{1.27} \OperatorTok{+}\StringTok{ }\FloatTok{1.88} \OperatorTok{+}\StringTok{ }\FloatTok{4.98}\NormalTok{))}
\NormalTok{g}
\end{Highlighting}
\end{Shaded}

\begin{verbatim}
## [1] -7.533333
\end{verbatim}

\begin{Shaded}
\begin{Highlighting}[]
\NormalTok{sp =}\StringTok{ }\FloatTok{11.72}
\NormalTok{n1 =}\StringTok{ }\DecValTok{21}
\NormalTok{n2 =}\StringTok{ }\DecValTok{34}
\NormalTok{n3 =}\StringTok{ }\DecValTok{20}
\NormalTok{n4 =}\StringTok{ }\DecValTok{25}
\NormalTok{n5 =}\StringTok{ }\DecValTok{7}
\NormalTok{n6 =}\StringTok{ }\DecValTok{82}
\NormalTok{n7 =}\StringTok{ }\DecValTok{48}

\NormalTok{standard_error =}\StringTok{ }\KeywordTok{sqrt}\NormalTok{(sp}\OperatorTok{^}\DecValTok{2} \OperatorTok{*}\StringTok{ }\NormalTok{((C1}\OperatorTok{^}\DecValTok{2}\OperatorTok{/}\NormalTok{n1) }\OperatorTok{+}\StringTok{ }\NormalTok{(C2}\OperatorTok{^}\DecValTok{2}\OperatorTok{/}\NormalTok{n2) }\OperatorTok{+}\StringTok{ }\NormalTok{(C3}\OperatorTok{^}\DecValTok{2}\OperatorTok{/}\NormalTok{n3) }\OperatorTok{+}\StringTok{ }\NormalTok{(C4}\OperatorTok{^}\DecValTok{2}\OperatorTok{/}\NormalTok{n4) }\OperatorTok{+}\StringTok{ }\NormalTok{(C5}\OperatorTok{^}\DecValTok{2}\OperatorTok{/}\NormalTok{n5) }\OperatorTok{+}\StringTok{ }\NormalTok{(C6}\OperatorTok{^}\DecValTok{2}\OperatorTok{/}\NormalTok{n6) }\OperatorTok{+}\StringTok{ }\NormalTok{(C7}\OperatorTok{^}\DecValTok{2}\OperatorTok{/}\NormalTok{n7)))}

\NormalTok{standard_error}
\end{Highlighting}
\end{Shaded}

\begin{verbatim}
## [1] 2.769041
\end{verbatim}

\begin{Shaded}
\begin{Highlighting}[]
\NormalTok{m =}\StringTok{ }\DecValTok{0}
\NormalTok{tm =}\StringTok{ }\NormalTok{(g}\OperatorTok{-}\NormalTok{m)}\OperatorTok{/}\NormalTok{standard_error}
\NormalTok{tm}
\end{Highlighting}
\end{Shaded}

\begin{verbatim}
## [1] -2.720557
\end{verbatim}

\subsubsection{\texorpdfstring{(c) (2 points) Find the resulting
\(p\)-value and state your
conclusion.}{(c) (2 points) Find the resulting p-value and state your conclusion.}}\label{c-2-points-find-the-resulting-p-value-and-state-your-conclusion.}

\begin{Shaded}
\begin{Highlighting}[]
\NormalTok{I =}\StringTok{ }\DecValTok{7}
\NormalTok{n =}\StringTok{ }\KeywordTok{sum}\NormalTok{(adder}\OperatorTok{$}\NormalTok{SampleSize) }
\NormalTok{df =}\StringTok{ }\NormalTok{n }\OperatorTok{-}\StringTok{ }\NormalTok{I}
\DecValTok{2} \OperatorTok{*}\StringTok{ }\NormalTok{(}\DecValTok{1} \OperatorTok{-}\StringTok{ }\KeywordTok{pt}\NormalTok{(}\KeywordTok{abs}\NormalTok{(tm),df))}
\end{Highlighting}
\end{Shaded}

\begin{verbatim}
## [1] 0.007015257
\end{verbatim}

Here we reject the Null Hypothesis that the average of the mean relative
head lengths for snakes on the Swedish mainland is equal to the average
of the mean relative head lengths for snakes on islands in the Baltic
Sea at significance level alpha = .05 in favor of the alternative
hypothesis that the average of the mean relative head lengths for snakes
on the Swedish mainland is not equal to the average of the mean relative
head lengths for snakes on islands in the Baltic Sea as p value is less
than alpha.

\subsection{\texorpdfstring{Question 3 (Modified from \emph{Sleuth}
6.21)}{Question 3 (Modified from Sleuth 6.21)}}\label{question-3-modified-from-sleuth-6.21}

Reconsider the education and future income data from your last homework
(data: \texttt{ex0525}). Find \(p\)-values and 95\% confidence intervals
for the difference in means for all pairs of education groups in the
following ways:

\subsubsection{(a) (2 points) Using the Tukey-Kramer
procedure.}\label{a-2-points-using-the-tukey-kramer-procedure.}

\begin{Shaded}
\begin{Highlighting}[]
\CommentTok{#View(ex0525)}

\NormalTok{Handicap_Mod <-}\StringTok{ }\KeywordTok{lm}\NormalTok{( }\DataTypeTok{formula =}\NormalTok{Income2005 }\OperatorTok{~}\StringTok{ }\NormalTok{Educ, }\DataTypeTok{data =}\NormalTok{ ex0525)}
\NormalTok{Handicap_Mod}
\end{Highlighting}
\end{Shaded}

\begin{verbatim}
## 
## Call:
## lm(formula = Income2005 ~ Educ, data = ex0525)
## 
## Coefficients:
## (Intercept)    Educ13-15       Educ16      Educ<12      Educ>16  
##       36865         8011        33132        -8563        39991
\end{verbatim}

\begin{Shaded}
\begin{Highlighting}[]
\NormalTok{Income_Mod <-}\StringTok{ }\KeywordTok{lm}\NormalTok{(Income2005 }\OperatorTok{~}\StringTok{ }\NormalTok{Educ , }\DataTypeTok{data =}\NormalTok{ ex0525)}
\NormalTok{Income_Mod}
\end{Highlighting}
\end{Shaded}

\begin{verbatim}
## 
## Call:
## lm(formula = Income2005 ~ Educ, data = ex0525)
## 
## Coefficients:
## (Intercept)    Educ13-15       Educ16      Educ<12      Educ>16  
##       36865         8011        33132        -8563        39991
\end{verbatim}

\begin{Shaded}
\begin{Highlighting}[]
\NormalTok{Income_Tukey <-}\StringTok{ }\KeywordTok{glht}\NormalTok{(Income_Mod, }\DataTypeTok{linfct =} \KeywordTok{mcp}\NormalTok{(}\DataTypeTok{Educ =} \StringTok{"Tukey"}\NormalTok{))}
\KeywordTok{summary}\NormalTok{(Income_Tukey)}
\end{Highlighting}
\end{Shaded}

\begin{verbatim}
## 
##   Simultaneous Tests for General Linear Hypotheses
## 
## Multiple Comparisons of Means: Tukey Contrasts
## 
## 
## Fit: lm(formula = Income2005 ~ Educ, data = ex0525)
## 
## Linear Hypotheses:
##                  Estimate Std. Error t value Pr(>|t|)    
## 13-15 - 12 == 0      8011       2201   3.639 0.002407 ** 
## 16 - 12 == 0        33132       2571  12.885  < 1e-04 ***
## <12 - 12 == 0       -8563       4000  -2.141 0.192917    
## >16 - 12 == 0       39991       2649  15.098  < 1e-04 ***
## 16 - 13-15 == 0     25121       2774   9.058  < 1e-04 ***
## <12 - 13-15 == 0   -16574       4133  -4.010 0.000556 ***
## >16 - 13-15 == 0    31980       2846  11.239  < 1e-04 ***
## <12 - 16 == 0      -41696       4341  -9.604  < 1e-04 ***
## >16 - 16 == 0        6858       3140   2.184 0.176474    
## >16 - <12 == 0      48554       4388  11.066  < 1e-04 ***
## ---
## Signif. codes:  0 '***' 0.001 '**' 0.01 '*' 0.05 '.' 0.1 ' ' 1
## (Adjusted p values reported -- single-step method)
\end{verbatim}

\begin{Shaded}
\begin{Highlighting}[]
\KeywordTok{confint}\NormalTok{(Income_Tukey)}
\end{Highlighting}
\end{Shaded}

\begin{verbatim}
## 
##   Simultaneous Confidence Intervals
## 
## Multiple Comparisons of Means: Tukey Contrasts
## 
## 
## Fit: lm(formula = Income2005 ~ Educ, data = ex0525)
## 
## Quantile = 2.7046
## 95% family-wise confidence level
##  
## 
## Linear Hypotheses:
##                  Estimate    lwr         upr        
## 13-15 - 12 == 0    8011.0607   2057.5359  13964.5855
## 16 - 12 == 0      33132.0768  26177.6652  40086.4885
## <12 - 12 == 0     -8563.4475 -19382.0985   2255.2034
## >16 - 12 == 0     39990.5665  32826.5062  47154.6268
## 16 - 13-15 == 0   25121.0161  17619.7682  32622.2640
## <12 - 13-15 == 0 -16574.5083 -27752.5263  -5396.4902
## >16 - 13-15 == 0  31979.5058  24283.4913  39675.5203
## <12 - 16 == 0    -41695.5244 -53437.2224 -29953.8264
## >16 - 16 == 0      6858.4897  -1635.5096  15352.4889
## >16 - <12 == 0    48554.0140  36686.9424  60421.0857
\end{verbatim}

\subsubsection{(b) (2 points) Without adjusting for multiple
comparisons.}\label{b-2-points-without-adjusting-for-multiple-comparisons.}

\begin{Shaded}
\begin{Highlighting}[]
\KeywordTok{summary}\NormalTok{(Income_Tukey, }\DataTypeTok{test=}\KeywordTok{adjusted}\NormalTok{(}\StringTok{"none"}\NormalTok{))}
\end{Highlighting}
\end{Shaded}

\begin{verbatim}
## 
##   Simultaneous Tests for General Linear Hypotheses
## 
## Multiple Comparisons of Means: Tukey Contrasts
## 
## 
## Fit: lm(formula = Income2005 ~ Educ, data = ex0525)
## 
## Linear Hypotheses:
##                  Estimate Std. Error t value Pr(>|t|)    
## 13-15 - 12 == 0      8011       2201   3.639 0.000279 ***
## 16 - 12 == 0        33132       2571  12.885  < 2e-16 ***
## <12 - 12 == 0       -8563       4000  -2.141 0.032380 *  
## >16 - 12 == 0       39991       2649  15.098  < 2e-16 ***
## 16 - 13-15 == 0     25121       2774   9.058  < 2e-16 ***
## <12 - 13-15 == 0   -16574       4133  -4.010 6.23e-05 ***
## >16 - 13-15 == 0    31980       2846  11.239  < 2e-16 ***
## <12 - 16 == 0      -41696       4341  -9.604  < 2e-16 ***
## >16 - 16 == 0        6858       3140   2.184 0.029062 *  
## >16 - <12 == 0      48554       4388  11.066  < 2e-16 ***
## ---
## Signif. codes:  0 '***' 0.001 '**' 0.01 '*' 0.05 '.' 0.1 ' ' 1
## (Adjusted p values reported -- none method)
\end{verbatim}

\begin{Shaded}
\begin{Highlighting}[]
\KeywordTok{confint}\NormalTok{(Income_Tukey, }\DataTypeTok{calpha =} \KeywordTok{univariate_calpha}\NormalTok{())}
\end{Highlighting}
\end{Shaded}

\begin{verbatim}
## 
##   Simultaneous Confidence Intervals
## 
## Multiple Comparisons of Means: Tukey Contrasts
## 
## 
## Fit: lm(formula = Income2005 ~ Educ, data = ex0525)
## 
## Quantile = 1.9609
## 95% confidence level
##  
## 
## Linear Hypotheses:
##                  Estimate    lwr         upr        
## 13-15 - 12 == 0    8011.0607   3694.7196  12327.4018
## 16 - 12 == 0      33132.0768  28090.0868  38174.0669
## <12 - 12 == 0     -8563.4475 -16407.0342   -719.8609
## >16 - 12 == 0     39990.5665  34796.5799  45184.5531
## 16 - 13-15 == 0   25121.0161  19682.5665  30559.4657
## <12 - 13-15 == 0 -16574.5083 -24678.6382  -8470.3783
## >16 - 13-15 == 0  31979.5058  26399.8492  37559.1623
## <12 - 16 == 0    -41695.5244 -50208.3257 -33182.7231
## >16 - 16 == 0      6858.4897    700.2894  13016.6899
## >16 - <12 == 0    48554.0140  39950.3161  57157.7120
\end{verbatim}

\subsubsection{(c) (3 points) What do you notice by comparing these two
methods? Discuss differences in which tests are significant, how wide
the confidence intervals are, and which confidence intervals contain
0.}\label{c-3-points-what-do-you-notice-by-comparing-these-two-methods-discuss-differences-in-which-tests-are-significant-how-wide-the-confidence-intervals-are-and-which-confidence-intervals-contain-0.}

We can say that, without adjustment there is all comparison (10)
significant. But when we consider Tukey-Kramer procedure that is with
adjustment there is 2 less significant comparisons.

We can also observe that in both the methods the point estimates are the
same (as we expect), but the lower and upper bounds are changing,
showing that the half-widths of the confidence intervals are changing.

Hence (\textgreater{}16 - 16) and (\textless{}12 - 12) comparisons
contain 0 in their confidence interval.

\subsubsection{\texorpdfstring{(d) (3 points) Use the Dunnett procedure
to compare every other group to the group with 12 years of education.
Look at both the \(p\)-values and confidence intervals. Which group
means apparently differ from the mean for those with 12 years of
education?}{(d) (3 points) Use the Dunnett procedure to compare every other group to the group with 12 years of education. Look at both the p-values and confidence intervals. Which group means apparently differ from the mean for those with 12 years of education?}}\label{d-3-points-use-the-dunnett-procedure-to-compare-every-other-group-to-the-group-with-12-years-of-education.-look-at-both-the-p-values-and-confidence-intervals.-which-group-means-apparently-differ-from-the-mean-for-those-with-12-years-of-education}

\begin{Shaded}
\begin{Highlighting}[]
\KeywordTok{with}\NormalTok{(ex0525, }\KeywordTok{levels}\NormalTok{(Educ)) }\CommentTok{# print levels before re-leveling}
\end{Highlighting}
\end{Shaded}

\begin{verbatim}
## [1] "12"    "13-15" "16"    "<12"   ">16"
\end{verbatim}

\begin{Shaded}
\begin{Highlighting}[]
\NormalTok{ex0525}\OperatorTok{$}\NormalTok{Educ <-}\StringTok{ }\KeywordTok{with}\NormalTok{(ex0525, }\KeywordTok{relevel}\NormalTok{(Educ, }\DataTypeTok{ref =} \StringTok{"12"}\NormalTok{))}
\KeywordTok{with}\NormalTok{(ex0525, }\KeywordTok{levels}\NormalTok{(Educ)) }\CommentTok{# print levels after re-leveling}
\end{Highlighting}
\end{Shaded}

\begin{verbatim}
## [1] "12"    "13-15" "16"    "<12"   ">16"
\end{verbatim}

\begin{Shaded}
\begin{Highlighting}[]
\NormalTok{Income_Mod <-}\StringTok{ }\KeywordTok{lm}\NormalTok{(Income2005 }\OperatorTok{~}\StringTok{ }\NormalTok{Educ, }\DataTypeTok{data =}\NormalTok{ ex0525)}
\CommentTok{#Income_Mod}

\NormalTok{Income_Dunnett <-}\StringTok{ }\KeywordTok{glht}\NormalTok{(Income_Mod, }\DataTypeTok{linfct =} \KeywordTok{mcp}\NormalTok{(}\DataTypeTok{Educ =} \StringTok{"Dunnett"}\NormalTok{))}
\KeywordTok{summary}\NormalTok{(Income_Dunnett)}
\end{Highlighting}
\end{Shaded}

\begin{verbatim}
## 
##   Simultaneous Tests for General Linear Hypotheses
## 
## Multiple Comparisons of Means: Dunnett Contrasts
## 
## 
## Fit: lm(formula = Income2005 ~ Educ, data = ex0525)
## 
## Linear Hypotheses:
##                 Estimate Std. Error t value Pr(>|t|)    
## 13-15 - 12 == 0     8011       2201   3.639   0.0011 ** 
## 16 - 12 == 0       33132       2571  12.885   <1e-04 ***
## <12 - 12 == 0      -8563       4000  -2.141   0.1180    
## >16 - 12 == 0      39991       2649  15.098   <1e-04 ***
## ---
## Signif. codes:  0 '***' 0.001 '**' 0.01 '*' 0.05 '.' 0.1 ' ' 1
## (Adjusted p values reported -- single-step method)
\end{verbatim}

\begin{Shaded}
\begin{Highlighting}[]
\KeywordTok{confint}\NormalTok{(Income_Dunnett)}
\end{Highlighting}
\end{Shaded}

\begin{verbatim}
## 
##   Simultaneous Confidence Intervals
## 
## Multiple Comparisons of Means: Dunnett Contrasts
## 
## 
## Fit: lm(formula = Income2005 ~ Educ, data = ex0525)
## 
## Quantile = 2.4808
## 95% family-wise confidence level
##  
## 
## Linear Hypotheses:
##                 Estimate    lwr         upr        
## 13-15 - 12 == 0   8011.0607   2550.2474  13471.8740
## 16 - 12 == 0     33132.0768  26753.2097  39510.9440
## <12 - 12 == 0    -8563.4475 -18486.7510   1359.8559
## >16 - 12 == 0    39990.5665  33419.4011  46561.7318
\end{verbatim}

We can see that all the group mean except \textless{}12 years of
education differe significantly from the group mean of 12 years of
education. Here, there is evidence that \textless{}12 and 12 years of
education have means that are same. All other groups have \(p\)-values
\textless{} 0.05 and confidence intervals does not contain 0)

\subsubsection{(e) (2 points) Taking all of these tests together, what
general statements would you make about the relationship between
Education and
Income?}\label{e-2-points-taking-all-of-these-tests-together-what-general-statements-would-you-make-about-the-relationship-between-education-and-income}

We can see that there's no evidence for the groups \textless{}12 - 12
and \textgreater{}16 - 16 that any of the groups have means that are
different from the other groups. We observe that (all \(p\)-values
\textgreater{} 0.05, all confidence intervals contain 0).


\end{document}
